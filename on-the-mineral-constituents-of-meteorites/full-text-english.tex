\documentclass[a4paper, 12pt, oneside]{article}
\usepackage[utf8]{inputenc}
\usepackage{fouriernc}
\usepackage{csquotes}
\usepackage{booktabs}
\usepackage{url}
\usepackage{graphicx}
\setlength{\emergencystretch}{15pt}
\graphicspath{ {./figures/} }
\usepackage[figurename=]{caption}
\usepackage{fancyhdr}
\usepackage{amssymb}
\usepackage{array}
\usepackage{float}
\usepackage{imakeidx}
\usepackage{qtree}
\renewcommand{\listfigurename}{List of Plates}
\makeindex[columns=2, title=Alphabetical Index, intoc]
\begin{document}
\begin{titlepage} % Suppresses headers and footers on the title page
	\centering % Centre everything on the title page
	\scshape % Use small caps for all text on the title page

	%------------------------------------------------
	%	Title
	%------------------------------------------------
	
	\rule{\textwidth}{1.6pt}\vspace*{-\baselineskip}\vspace*{2pt} % Thick horizontal rule
	\rule{\textwidth}{0.4pt} % Thin horizontal rule
	
	\vspace{0.75\baselineskip} % Whitespace above the title

    {\LARGE On the Mineral Constituents\\ of Meteorites.} % Title
	
	\vspace{0.75\baselineskip} % Whitespace below the title
	
	\rule{\textwidth}{0.4pt}\vspace*{-\baselineskip}\vspace{3.2pt} % Thin horizontal rule
	\rule{\textwidth}{1.6pt} % Thick horizontal rule
	
	\vspace{1\baselineskip} % Whitespace after the title block
	
	%------------------------------------------------
	%	Subtitle
	%------------------------------------------------
	
	{By Nevil Story-Maskelyne, MA.\\ \small Professor of Mineralogy, Oxford, and Keeper of the Mineral Department, British Museum.} % Subtitle or further description
	
	\vspace*{1\baselineskip} % Whitespace under the subtitle
	
	%------------------------------------------------
	%	Editor(s)
	%------------------------------------------------
	
	\vspace{1\baselineskip} % Whitespace before the editors

    %------------------------------------------------
	%	Cover photo
	%------------------------------------------------
	
	%\includegraphics[scale=1]{cover}
	
	%------------------------------------------------
	%	Publisher
	%------------------------------------------------
		
	\vspace*{\fill}% Whitespace under the publisher logo
	
	October 9, 1869. % Publication year
	
	{\small Royal Society of London. } % Publisher

	\vspace{1\baselineskip} % Whitespace under the publisher logo

    Internet Archive Online Edition  % Publication year
	
	{\small Attribution NonCommercial ShareAlike 4.0 International } % Publisher
\end{titlepage}
\setlength{\parskip}{1mm plus1mm minus1mm}
\setcounter{tocdepth}{3}
\setcounter{secnumdepth}{3}
\tableofcontents
\clearpage
\section{The applications of the Microscope in the investigation of Meteorites.}
\paragraph{}
The mineralogical investigation of a meteoric stone presents difficulties very similar to those which have hitherto rendered the analyses and descriptions of many of the finer-grained terrestrial rocks unsatisfactory; for a meteoric stone is in fact a fragment of a rock, though formed under conditions in some respects widely differing from those which have produced the rocks of our globe.

The difficulties alluded to arise from the minute size and imperfectly developed crystallisation of the mineral constituents alike of the rock and the aerolite; and they have in general baffled the efforts of the chemist on the one hand to effect their separate analyses, and of the crystallographer on the other hand to determine the forms of these constituents. The chemist indeed has endeavoured to overcome the difficulty by attempting a chemical separation of the constituent minerals of these fine-grained mixtures into one group of such as are soluble and another group of those which are insoluble in acids, and then treating the numbers obtained from the analyses of these groups by the light of theoretical considerations founded on the formula and properties of known minerals. This method is necessarily only an approximative one. Even granting that by its means we could divide a rock into two classes of ingredients, which we cannot in fact accurately do, there remains the question of how to separate from each other the mingled minerals in, for instance, its insoluble portion.

But the great interest that attaches to whatever may throw light on the history of aerolitic rocks seemed to render it very desirable that some more reliable method should be sought for their investigation. With this end in view, and also with the purpose of basing on such an investigation a scientific classification of the now very extensive collection of aerolites in the British Museum, I some six years ago commenced a systematic examination of these bodies by the microscope. While the meteorites were being cut in order to show their polished surfaces, a small fragment of the portion detached was fastened by its flat side to a strip of glass and carefully worked down to the utmost tenuity. The transparent section thus formed was then examined in the microscope. The results to be obtained by the study of such sections may be divided into such as are structural, throwing light on the physical conditions under which the meteorite was formed, and such as are mineralogical and concern purely the particular minerals that are the ingredients of the stone. From the former class, we learn that a meteorite has had a history; that it has undergone change subsequently to its first consolidation in its present form. The crystalline character of all the constituent minerals; the fissures at one time formed, then filled and then, in many cases, broken across and `heaved' and filled again, like some mineral lode; the `chondritic' structure that G. Rose has illustrated\footnote{Abhandl. der Königl. Akademie der Wissensch. Berlin, 1864, p. 84.}; the fracture, in at least one meteorite, of the spherular `chondra,' which have been split and severed and recemented into a compact mass, --- these are among the many facts imprinted on a meteorite which are so many records belonging to its history, and which by the aid of the microscope we may read and interpret. But to found a classification on the structural characters of meteorites is not the same thing as to arrange them according to their mineralogical composition. The two must be combined for a philosophical arrangement.

I propose dealing with the mineralogical side of the problem in the present memoir, and to recur hereafter to the structural composition of meteorites, when the nature of their ingredient minerals shall have been rendered clear.

The general features of the microscopic sections of certain meteorites were described by me in the years 1863-64; and the examination in this way has been extended to above 140 distinct aerolites. The crystallography, however, of the numerous crystals seen in such a microscopic section is almost hopelessly difficult. In cases where crystallographic directions are indicated by cleavage-planes or by the `traces' on the section of determinable crystal faces, some conclusion as to the symmetry and system of the crystal can be drawn from the directions of its optical principal sections as indicated by light polarised in a known plane. And occasionally a section is met with so nearly parallel to one of the important faces of the crystal as to allow some reliance to be placed on the angles of its bounding planes as measured by a delicate eyepiece goniometer. A long series of measurements and determinations of the directions parallel to the principal sections in the crystals met with in these microscopic slides has convinced me that, however useful the microscope may be in revealing the structure of a meteorite and helping to determine its place in a collection of such bodies classified according to their physical constitution, it is only partially of use in determining the mineralogical character of the constituents. But if the applications of polarised light and the eyepiece goniometer are thus limited, the microscope has another function to perform in such an investigation; for, from the carefully bruised \emph{débris} of particular meteorites selected for the frequent recurrence in them of recognisable minerals, and for the magnitude of the grains of these, one is able to pick out under the microscope the distinct particles of each such mineral.

Such particles occasionally offer cleavage-planes, or even a crystal face or two, to the goniometer. In a very few cases crystals have been found sufficiently complete to lead to a reliable crystallographic result.\footnote{The crystals of Anorthite in the Juvinas Meteorite were thus measured in the British Museum by my late colleague, Professor V. von Lang, Sitzungsber. Akad. der Wissensch. Wien, 1867.} The chief advantage of this method is, however, that it affords the means of analysing the separate minerals selected by it; but even thus the difficulties are considerable. The quantity of material at one's disposal amounts generally to a very few grains, and of these but a small proportion consists of the desired mineral; in fact half a gramme is as much as one is generally able to obtain in a state of purity.

The analysis of so small an amount of a silicate is a difficult problem. To break up the silicate by fusion with alkalis implies the introduction of foreign and non-volatile ingredients; to estimate the silica by its loss on treating the mineral by ammonium or hydrogen fluoride were to lose what, in the analysis of such small quantities, is the necessary check afforded by the summing of the percentages of its constituents.

To distil the silica in the form of silicium fluoride, and then to determine it either as silica or as potassium fluosilicate, suggested itself as a means by which this check might be secured. A long series of experiments, undertaken in order to ascertain the best process for thus determining the silica, has resulted in a completely successful application of the method; and by it so small an amount as two-tenths of a gramme of enstatite or of augite has been analysed with a satisfactory result in the Laboratory at the British Museum.
\clearpage
\section{On a new Method of analysing Silicates that do not gelatinise with Acids.}
\paragraph{}
The method adopted for the analysis of silicates in small quantity, to which reference has already been made, was the following.

Hydrogen fluoride formed from picked fluor-spar was conducted into water. When the saturation had reached the point at which the liquid gave off fumes, the acid solution was treated in a platinum dish with potassium fluoride so long as any precipitate was formed. After decantation the acid was distilled, the first and last portions being omitted, and the distilled acid preserved in a platinum bottle. A leaden bottle, even when lined with pure gutta percha deposited from its solution in benzol, appears to be attacked by the acid.

A small platinum retort of a capacity of 30 cub. centims. (fig. 1) has fitted into it a tubulated stopper (fig. 2) reaching nearly to the bottom of the retort; a small tube (\emph{b}) enters the straight tube (\emph{k}) of the stopper (\emph{a}) at an angle above the neck of the retort, for the delivery of hydrogen. The straight tube can be stopped either by a small platinum stopper (fig. 3), or by a funnel of that metal (fig. 4) with a stopper (\emph{f}) at the top and a fine orifice at its lower extremity (\emph{h}).

In the side of the retort just below the neck a straight delivery-tube is fixed, which again fits into another platinum tube (fig. 5) that, after taking a curve into a vertical position, is enlarged into a long cylinder capable of passing nearly to the bottom of a test-tube. The test-tube, into which it is fitted by a cork, holds when properly charged 7.5 cub. centims. = 6.6 grammes of a strong solution of ammonia (of specific gravity = 0.88), corresponding to 2.03 of H\textsubscript{3}N, and a glass delivery-tube passes to the bottom of another test-tube, also containing a little of that alkali.

The mineral to be analysed is first powdered extremely carefully in an agate mortar; of this a quantity, that may be from 0.2 to 0.5 gramme, is introduced into the retort together with a small platinum ball. The tubulated stopper (fig. 2) is now introduced into its place and cemented by the aid of a little gutta-percha varnish, and by winding over its edge and that of the retort a strip of thin sheet india-rubber. The funnel, with its top closed, is now put into the tubulure of the stopper and filled with the hydrogen fluoride, which has a strength of about 32 per cent. of acid: it contains 1.12 grm. of pure hydrogen fluoride, capable of rendering gaseous 0.84 grm. of silicic acid (SiO\textsubscript{2}), and of neutralising 0.95 of ammonia (H\textsubscript{3}N). The acid is next admitted to the silicate, and the funnel removed to give place to the small stopper, which, and the joint of the platinum delivery-tubes, are now sealed with gutta-percha varnish. The apparatus, as prepared for use, is represented in fig. 6.\footnote{The scale of figs. 1 to 5 is one-half the actual size of the apparatus; that of fig. 6 is one-third the actual size.} Pure dry hydrogen is next allowed slowly to traverse the whole apparatus, and the retort is placed in a water-bath at 100° C. for two hours, and occasionally shaken to set the ball in motion. During this operation only a minute trace of silicium difluoride comes over.

The retort is next transferred to a bath of paraffin and carefully heated in it. At first hydrogen fluoride comes over; and at this point of the process the flow of the hydrogen requires a little attention. At about 132° C., with the silicates described in this memoir, the silica first becomes visible in fine flocks in the ammonia solution, and in another minute the whole is cloudy. In from five to ten minutes the temperature has risen to 142°-145°, and so much of the fluoride has come over that the contents of the tube are of a semi-solid consistency, and nearly the whole of it has in fact passed over. The temperature is allowed to increase to 150°, and the retort then permitted to cool. The process is repeated by introducing a fresh charge of hydrogen fluoride into the retort and of ammonia into the test-tube, and again heating in the paraffin bath. If the quantity of the silicate taken be not more than 0.2 grm., twice charging the retort is sufficient; if it amounts to 0.5, three or four repetitions of the process are required. The process must in fact be repeated so long as any fresh flocks of silica can be seen to form in the ammonia tube. Finally, 0.75 cub. centim. of sulphuric acid are introduced into the retort, and the temperature raised to 160°, the stream of hydrogen being continued as before.\footnote{In a series of analyses made with a view to determine the degree of energy with which the acid attacks various silicates and the forms of silica itself, it was found that the first turbidity of the ammonia will, if sufficient time be allowed, commence at 120°C. At this temperature to 121°C. twenty-three minutes were requisite for the bulk of the silicium compound to come over from the retort. It was not found, however, that the action of the acid was more energetic on one silicate than on any other, or on quartz.}

The several ammoniacal charges of the ammonia tube are now brought together into a platinum dish with all the washings from the test-tubes and the connecting tubes; and these are now slowly evaporated in a water-bath with continual stirring.

At a certain point of the evaporation, just before the solution becomes neutral, and the ammonium fluoride begins to become acid, all the silica in the dish is dissolved by the fluoride. The process is gradual, but the moment is easily determined when it is complete. Then the dish being removed from the water-bath, potassium chloride is added in slight excess; and absolute alcohol equal in bulk to the liquid in the platinum vessel is poured on it. Potassium fluosilicate precipitates, and after standing twenty-four hours it is filtered and washed with a mixture of equal volumes of absolute alcohol and water, and then dried and weighed. The results are accurate.

In the platinum retort are the bases, in the form of sulphates, the treatment of which calls for no further remark.

A specimen of diopside, pulverised and analysed by the method here described, yielded 53.46 per cent.; on treatment by fusion with potassium and sodium carbonate, it gave in two analyses 53.51 and 53.54 per cent. of silicic acid.
\clearpage
\section{The Busti Aerolite of 1852.}
\paragraph{}
Among the meteorites with ingredients sufficiently large in the grain to offer an opportunity for isolating and determining their constituent minerals, is a stone that fell in India on the 2\textsuperscript{nd} of December, 1852, near the station named Busti, situated about halfway between Goruckpur on the east and Fyzabad on the west, and consequently some 45 miles from Goruckpur, and nearly in 26° 49$^{\prime}$ north latitude and 82° 42$^{\prime}$ east longitude.

For the account of the circumstances attending the fall of this meteorite I am indebted to Mr. George Osborne, at that time Resident at the Busti Station, to whose care science owes the preservation of the stone that fell there. He presented it to the East-India Company, and for several years it stood in the Library of the India House. It was presented to the British Museum when Lord Halifax was Secretary of State for India, by the Secretary of State in Council. Mr. Osborne describes the fall as having taken place at 10h 10m AM, announcing itself by a sudden explosion much louder and of a more detonating character than an ordinary thunderclap, increasing in intensity towards its termination. There was no trace of cloud in the sky, and the report lasted for a time that Mr. Osborne estimated at from three to five minutes. At Busti the report was not accompanied by the effects of concussion, while at Goruckpur it shook the glass and doors in the houses, the sound appearing at the latter station to approach in a direction from WNW. At Busti it seemed to one facing the north to come from the zenith; and though heard so loudly at Busti, and apparently still more loudly at Goruckpur, it was not noticed at a station thirty miles west of Busti. The course of the stone was probably a north-easterly to a south-westerly one, the explosion that shattered it having occurred soon after it had passed the longitude of Goruckpur. The stones fell at a place six miles south of Busti, and Mr. Osborne obtained a small one weighing about three pounds. How many fell was not ascertained, but all the others have been lost sight of, Mr. Osborne having in vain endeavoured to obtain a second.

The aspect of the specimen of this aerolite which Mr. Osborne preserved is in many respects very similar to the stone that fell at Bishopville, in South Carolina, USA, on the 25\textsuperscript{th} of March, 1843. The form and actual size of the stone are represented in Plate 2, in which two views from opposite points are given, the orientation being shown by the position of the letters A, B, C, D in the two views. The crust which coated the larger part of the stone was exceptional in character. At the flat end this crust was of a dark yellowish brown, with a few yellowish-white porphyritic-looking patches where the brown crust was thinner. In the large hollow portion on one side, near to C in the lower view, a yellowish enamel mingled with a very dark grey enamel is also relieved by white markings like the white felspar of a porphyry. In other places these white or yellowish-white markings, with their angular but unsymmetrical outlines, are seen sharply contrasted with the black-grey enamel only. It is difficult to connect the outlines of these porphyry-like markings with those of crystals of any mineral underlying them. Over augite and enstatite alike, where they occur in this stone, the crust seems to be similar in its features. I can only suppose the natural hue of the crust, due to the fusion of the silicates, to be a pale yellow; but that the metallic nickeliferous iron, found here and there in grains of considerable size, has, during the fusion and dispersion of the outer parts of the stone in its career through the air, fused down with portions of the silicates into a dark and perhaps more fusible enamel that has mixed unevenly as it flowed over the less fluid glass of the silicate. The enamel, it should be added, has generally vesicular appearances when seen under a high power, as if gases had escaped from it during its fusion.

One end of the meteorite presented a remarkable feature; small round chestnut-brown spherules, coated with yellowish-white fused silicate, stood out from a well-defined nodule that was imbedded in this portion of the stone near N in the upper view. In these small spherules there might in one or two places be seen, with a lens, minute octahedral crystals with the lustre and colour of gold. These two minerals seemed scarcely to have been affected by the heat that fused the silicates in which they are imbedded, and which protected their surfaces from the action of the atmospheric oxygen; but in one or two cases the crust at these spots was rendered darker in colour by the influence of the nodule it covered.

It was indeed fortunate that this meteorite came in its entirety into the British Museum. A blow from a hammer (the too usual fate of Indian meteorites) would have scattered the contents of this nodule as dust, for its peculiarities were only visible from the outside on very careful inspection. A section was made so as to pass through this nodule; it was then seen to be definitely bounded by a black line, within which two distinct silicates could be detected, and the polished surface was dotted with the round spherules of the chestnut-brown mineral that has been alluded to. A representation of this section parallel to the line N is given in Plate 2.

The powder produced in the cutting of the meteorite and a few fragments of the separated portion, too small for distribution to other museums, were retained for chemical examination, and slides of the different minerals were worked from some of these fragments for the microscope. A small nodule of the metallic iron was also preserved for analysis. From the fragments the different minerals were picked out under the microscope, and among these a few specimens were found sufficiently complete to throw light on their crystallography.
\clearpage
\section{Oldhamite-Sulphide of Calcium.}
\paragraph{}
I gave the name of Oldhamite to this mineral in 1862, when it first attracted my attention, though I had not then the opportunity of properly investigating it.\footnote{I named it in compliment to Dr. Oldham, Director of the Indian Geological Survey, who in that year acted on behalf of the Asiatic Society of Calcutta on the occasion of that Society giving, in the most liberal spirit, to the British Museum large portions of several important aerolites that had fallen on Indian territory, and were preserved in its Museum.}

Oldhamite is a pale chestnut-brown and, where pure, transparent mineral occurring in the Busti aerolite, and apparently also sparsely in that of Bishopville, in small nearly round spherules imbedded in enstatite or augite, or in a mixture of both. The outer surface of the spherules is generally partly coated by calcium sulphate, the result of the oxidation of the sulphide. When the adhering silicate and this crust have been removed, the mineral is readily cleaved in three directions. The mean of nearly 200 measurements of the normal angles of these cleavage-planes gave 89° 57$^{\prime}$. That this angle is really 90° and the mineral cubic in its system, rendered probable by the equal facility of the three cleavages, is placed beyond doubt by the fact that transparent sections made along either plane, when examined by polarised light, afford no indications whatever of double refraction. The hardness of the mineral is nearly 4; its density is 2.58. Boiled with water it breaks up, yielding a bright-yellow solution of calcium polysulphides and an insoluble residue. With acids it readily dissolves with evolution of hydrogen sulphide and deposition of sulphur.

The small amount of the mineral at command for analysis, the desirability of excluding as far as possible during the process all non-volatile materials from admixture with the constituents of the meteorite, and, as it afterwards proved, the unnecessary precaution of not using any reagent that might destroy or prevent the subsequent selection of the microscopic octahedra that have been alluded to, seemed to render a special method of analysis necessary.

Experimental analyses of calcium sulphide, formed by passing first hydrogen and subsequently hydrogen sulphide over caustic lime ignited in a glass tube, led to the employment of the following method.\footnote{This calcium sulphide, in colour and in all but its want of cohesion and crystalline characters, resembles Oldhamite. Thus formed it contained 44.3 per cent. (it should contain 44.44) of sulphur. Formed without the previous treatment with hydrogen it contained only 39.5 per cent.}

0.4696 grm. of the mineral, dried over sulphuric acid for thirty-six hours, were placed in a small flask with a stoppered funnel filled with previously boiled but cold water. Hydrogen, purified by traversing a solution of lead acetate and a U-tube containing glass moistened by that liquid, was then passed by a tube through the cork into the flask, whence it was conducted by a delivery-tube into a solution of 3 grms. of pure silver nitrate, and thence, finally, by a second tube through a test-tube similarly charged. First a little of the boiled water and strong hydrogen chloride were introduced by the funnel tube, and the hydrogen sulphide formed was swept out by the hydrogen as fast as it was generated. The action having ceased, the flask was carefully heated for six or seven hours, when the hydrogen no longer discoloured lead test-paper. The silver sulphide was then thrown on a weighed filter, washed with ammonia and water, dried at 100°, and weighed. Some sulphur separated in the flask; this, together with the undissolved silicates and golden-yellow crystals, was collected on a dry filter and weighed, and the sulphur was then removed by carbon disulphide, and its amount determined by the difference of weight. The yellow crystals were picked out from the undissolved residue, and there remained the silicates that had surrounded or been entangled with the spherules of Oldhamite.

The results of the first analysis are given below as No. 1; those of a second, in which 0.5061 grm. of the mineral in rather large spherules were taken, are given as No. 2.

Table 1.

|1.|2.  
Undissolved silicate|---, ---, 7.644|---, 8.456  
Octahedra (Osbornite)|---, ---, 0.277|---, 0.297  
Dissolved silicate (enstatite)|{SiO\textsubscript{3} 1.159, Mg 0.366}, 1.525|{0.875, 0.276}, 1.151  
Residuary magnesium (as magnesium monosulphide)|{Mg 1.257, S 1.676}, 2.933|{1.229, 1.638}, 2.867  
Calcium monosulphide|{S 35.888, Ca 44.860}, 80.748|{35.227, 44.034}, 79.261  
Calcium sulphate (gypsum)|{Ca 0.830, SO\textsubscript{4} 1.933, H\textsubscript{2}O 0.747}, 3.570|{0.856, 2.054, 0.770}, 3.680  
Calcium carbonate|{Ca 1.241, (CO\textsubscript{2} 1.861)}, 3.102|  
Iron sulphide (Troilite)||{Fe 1.288, S 0.736}, 2.024  
Iron (probably metallic)|{Fe 0.507}|---, 0.261  

The dissolved silicate, as will be seen in the sequel, is most likely to be enstatite, that mineral being the more soluble of the two silicates in which the Oldhamite is imbedded. That a portion of the magnesium is present as sulphide is a necessary conclusion from the proportions of sulphur and of metal in the spherules. I have assumed the sulphuric acid to correspond to the ordinary hydrated calcium sulphate (gypsum), and the residuary calcium in analysis 1 to be present as calcium carbonate. The result of this interpretation of the above analyses is that, if we deduct the silicates encrusting the Oldhamite together with the Osbornite and the iron that the spherules contain, we have the following composition for these spherules and their oxidised coating.

Table 2.

|1.|2.  
Oldhamite|{Calcium monosulphide 80.748 89.369, Magnesium monosulphide 2.933 3.246}|{Calcium monosulphide 79.261 90.244, Magnesium monosulphide 2.867 3.264}  
Incrustation|{Gypsum 3.570 3.951, Calcium carbonate 3.102 3.434, Troilite --- ---}|{Troilite 2.024 2.303}  

The magnesium sulphide may be looked on either as a mechanically mixed ingredient or as a constituent of the Oldhamite.\footnote{In the second analysis the deficiency in the percentage may have been due to some small error in the determination of the lime; and any such small error, where the quantity of the mineral disposable for the analysis is so minute, becomes magnified considerably on calculating the percentage results.}

The existence of either of these substances in a meteorite serves to prove that the conditions under which the ingredients of that rock came into their present form were very unlike those met with on the surface of our globe. Water and free oxygen must alike have been absent, or only present in inappreciable quantities; indeed the existence of the metallic iron in the state of minute division in which it so frequently occurs in meteorites would lead to a similar conclusion. But the evidence afforded by the aerolite of Busti seems, further, to point to a reducing agent having been present during the formation of its constituent minerals; while the crystalline structure of Oldhamite, and of the Osbornite next to be described, must certainly have been the result of fusion at an enormous temperature.

The detection of hydrogen by Professor Graham in meteoric iron tends to confirm the probability of the presence of a reducing agent among the conditions under which these meteoric minerals were formed.
\clearpage
\section{Osbornite.}
\paragraph{}
The golden-yellow microscopic octahedra that have been mentioned in the description of Oldhamite were furnished by the analysis of that mineral to the amount of only 0.0028 grm., the first analysis having yielded 0.0013, and the second 0.0015 grm. Yet even this minute weight, forming less than 0.3 per cent. of that of the Oldhamite, was divided between upwards of 150 crystals. These crystals were nevertheless capable of being measured by the goniometer.

This microscopic mineral I wish to name Osbornite in honour of Mr. Osborne, and in order to commemorate the important service that gentleman rendered to science in preserving and transmitting to London in its entirety the stone which his zeal saved at the time of its fall, and in recording all he could collect about the circumstances associated with that fall.

That the octahedra of Osbornite are ``regular'' octahedra will be apparent from the following results of their measurement. A supplementary lens applied to the object-glass of the telescope of the goniometer enables the observer to determine that position of the face of a crystal in which the illumination from a narrow slit in a distant window is at its maximum. In this way the angles between faces can be measured when the faces themselves are too small, or are too dull, or too much striated for use as reflectors of the image of the slit.

This approximate method of measuring the angles of so minute a crystal, when used with some crystals of Osbornite, gave for:

1 1 1, 1 1 1$^{\prime}$ a mean of fifteen measurements on two crystals 70° 27$^{\prime}$, Regular octahedron 70° 31$^{\prime}$  
1 1 1, 1 1$^{\prime}$ 1$^{\prime}$ a mean of nine measurements on two crystals 109° 31$^{\prime}$, Regular octahedron 109° 28$^{\prime}$  
1 1 1, 1 1$^{\prime}$ 1 a mean of six measurements on two crystals 69° 58$^{\prime}$  
1 1 1, 1 1 1$^{\prime}$ a mean of six measurements on a third crystal 70° 37$^{\prime}$

The analysis of this mineral presented a very difficult problem, the total amount available being too minute for any quantitative results to be expected from it. Moreover it was found that these little crystals resisted the action of the acids when in their integrity, and when crushed their minute size rendered the manipulation most difficult and the results uncertain. Boiled for a long while in nitric acid they were unchanged; and even hydrogen fluoride had no apparent action on them. They passed unscathed through a fusion of a small amount of Oldhamite with potassium-sodium carbonate; though when fused with potassium chlorate, a crystal of Osbornite entirely disappeared --- perhaps from its escaping notice.

The crystals are very brittle, and after being crushed the powder retains the beautiful golden colour of the surface, which is therefore intrinsic and not due to a tarnish; to which cause, however, a ruddy hue in two of the crystals may have been due.

A few of the crystals placed in a glass tube and ignited in a current of dry oxygen, underwent only an external oxidation. Similarly treated in a slow current of chlorine they were decomposed; but they sank into the glass before the decomposition was complete. An experiment was therefore made in which the crystals were supported in a cavity scooped into a small fragment of the thinnest Japanese porcelain. The chlorine was passed through the apparatus and the tube ignited previously to the introduction of the octahedra, in order to determine that the chlorine was without action on any part of the apparatus. This apparatus consisted of a small tube in which the porcelain splinter was placed; one end of it was drawn out so as to form a U tube of considerable length, surrounded at the bend by a freezing-mixture, and dipping ultimately into a tube of pure water.

The whole of the available material, except a few crystals reserved for the goniometer, was placed in the cavities of the porcelain splinter, and dry chlorine allowed slowly to traverse the tube. On the application of heat shortly below redness, a glow was seen to commence among the minute crystals, which, extending itself through the whole, lasted for a few seconds.

The crystals appeared to have somewhat increased in bulk; they still retained their forms, but their metallic lustre had left them, and their colour became of a pale honey yellow.

The tube had become slightly iridescent in front of the assayed mineral; the drawn-out portion of it contained a small amount of a white sublimate, and a slight fuming came with each chlorine bubble through the water.

The altered crystals on being exposed to the air soon began to deliquesce and assumed a pasty consistence; treated with water, they only partially dissolved and gave the solution an alkaline reaction. This solution gave no precipitate with ammonia, but yielded one to ammonium oxalate. The residue undissolved by this water dissolved, though neither with ease nor quite completely, in hydrogen chloride.

This acid solution gave with ammonia a slight precipitate, which was redissolved in acid and reprecipitated: it seemed to contain a little iron. Filtered from this, the solution further gave a very distinct precipitate on being treated with ammonium oxalate and kept for some time warm. Hydrogen disodium phosphate gave a very slight flocculent turbidity to the filtrate from that precipitate, but it had not the appearance of the magnesium salt.

The interior of the portion of the tube drawn out and kept cool at the bend by a cooling-mixture was lined by a slightly yellowish-white sublimate.

This white body was treated with hydrogen chloride, in which it was at first somewhat difficult of solution, and was added to the water into which the chlorine had been passed. That liquid, which was acid in its reaction, was evaporated down as a clear solution till it had become but a few drops, and barium chloride was added to it. A precipitate of barium sulphate was formed, which after some time was filtered off and weighed. It weighed 0.0008 grm. There can be no doubt, therefore, that Osbornite contains sulphur as an important constituent. The excess of barium was removed and ammonia added, which threw down a very decided flocculent nearly white precipitate. The filtrate from this precipitate left no visible residue on evaporation. The precipitate itself readily redissolved in hydrogen chloride, and was again reprecipitated by ammonia; it resembled alumina in appearance, but it proved to be entirely insoluble in potash. The experiment, more than once repeated, of redissolving and reprecipitating it, as before, invariably gave a body insoluble in potash.

Dissolved again in acid and all excess of the acid having been removed, the addition of sodium hyposulphite produced, on warming it, a white precipitate. Restored to its former condition of as nearly neutral a solution as possible, on being treated by an excess of potassium sulphate, it gave a white precipitate.

The insolubility of the ammonia-precipitate in caustic potash having pointed to the probability that either titanium or zirconium oxide was present, some preliminary comparative experiments were made which led to the following method.

Three small glass tubes, similar in all respects, were taken, and into one a portion of the precipitate formed by ammonia from the Osbornite was transferred; into a second a quantity to appearance rather less of titanium oxide, formed in a similar manner by solution in acid and reprecipitation by ammonia, was put; and the third was similarly charged with zirconium oxide, formed in as similar a manner as possible by passing chlorine over a heated mixture of zirconia and charcoal, and treating the sublimate as the mineral sublimate had been treated.\footnote{It is remarkable that in this experiment the same internal iridescence of the glass tube was observed as in the experiment with the mineral, and the chloride sublimed in the same manner and with the same appearance.}

Hydrogen chloride and water were next added in equal amounts to each tube. A minute bit of magnesium wire was dropped into each, and the changes were watched under the microscope with an inch objective. After a certain period black patches began to appear on the magnesium wire in the titanium oxide tube, and a bluish coloration could be faintly discerned: in the tube with the precipitate from the Osbornite the magnesium retained its silvery brightness entirely unstained, as did the wire in the zirconia tube. Repetition of the experiment confirmed the delicacy of this test for the presence of titanium oxide, a test tried by which the precipitate from the Osbornite failed invariably to show any evidence of the presence of that oxide. Phosphorus was carefully looked for, but ammonium molybdate failed to give any trace of an indication of its presence.

A negative test of the above kind would not afford sufficient ground for asserting the presence of zirconium; on the other hand, it would seem a fair presumption that titanium at least is not present in Osbornite. That the metal to which these reactions are attributable is not zirconium, however, may be affirmed with some certainty. Mr. Sorby has made the pyrognostic characters of this element a special study; and I gave that gentleman rather more than half of the minute amount I possessed of the precipitated oxide.

He examined it in a microscopic borax bead, and asserts that he failed to obtain the crystals characteristic of zirconium, and that the chief and probably the only constituent of this substance is titanic acid, as the crystalline deposit in the bead exactly accords with the very peculiar forms assumed under the same conditions by that oxide.

With so infinitesimal an amount of substance at one's disposal it seems impossible to investigate further the nature of this element. Even the methods of the spectrum-analysis are not yet reduced to a form available for determining the nature of an element of this group in so minute an amount. That it is not zirconium the evidence of so accurate an experimentalist as Mr. Sorby may be taken to prove; that it is titanium seems scarcely compatible with the comparative experiments I made with it and with titanium chloride. It is certain, however, that Osbornite consists of calcium, and what may be provisionally termed a titanoid clement, possibly titanium itself, with a trace of iron and combined with sulphur in some peculiarly stable form. This form can hardly be that of a combination of the sulphides of the metals merely; one cannot well conceive such a compound resisting the action of acids. The sulphides of the metals of this class, however, are little known; while those of calcium associated with oxygen in the form of what are termed oxysulphides need also further investigation.

The fact of the Osbornite crystals being met with occasionally in the variety of augite, which will be presently described, as an ingredient of this meteorite, and which is for the most part confined to that nodule of the meteorite in which the Oldhamite occurs, suggested a search in that silicate for an oxide corresponding to what had been found in the Osbornite. The precipitates thrown down by ammonia from the acid solution of the bases in the different analyses of that augite were therefore brought together and examined. They contained some ferric oxide; but this was associated with a small amount of a colourless oxide entirely insoluble in potash, which, when tried by the tests that had been employed with the Osbornite precipitate, gave exactly the same results as these had given. The dichroism of this augite is very marked; and on looking through one of its faces (the face 0 1 0) a tint (like that of the bluish anatase from Brazil) is seen that appears to be due to certain minute interlaminated layers permeating the augite, but which require the microscope for their exhibition.

In whatever manner, whether as a constituent base in the augite itself, or as a foreign body interlaminated with it in the direction of the planes 0 0 1, 0 1 0, the evidence of the augite goes to confirm that of the analysis of the Osbornite, namely, that a metal nearly related to, if it be not titanium is present in both minerals. Possibly the minute interlaminated mineral alluded to may consist of Osbornite of sufficient thinness to be transparent, and to give the colour alluded to. It is remarkable that the metallic sheen on the plane 1 0 0 of the augite is of a golden yellow by reflected light, and exhibits the bluish tint by transmitted light.

It may not be out of place here to call attention to a singular golden-yellow incrustation, cubic in the form of its particles, obtained by Professor Mallet, of Alabama University, USA, by heating metallic zirconium to an intense heat in a furnace with lime and aluminium. These crystals were not analysed, but it is not impossible that sulphur from the fuel might have supplied that ingredient, and that these crystals were in their nature analogous to those revealed to us in this meteorite, for like the Osbornite crystals they were not attacked by the strongest acids (see American Journal of Science, Series 2, vol. 28. 1859, p. 346).
\clearpage
\section{The Augitic Constituent of the Busti Aerolite.}
\paragraph{}
Associated with the spherules of calcium sulphide that have been described as occurring in a nodule in this aerolite, and also less plentifully distributed through the rest of its mass, is the silicate, to which allusions have already been made as a variety of augite, and as containing traces of an element with some of the chemical characteristics of titanium. This silicate is of a pale violet-grey colour, intimately mixed in the form of crystalline grains with another silicate presently to be described. These crystalline lilac-grey grains, when isolated as much as possible from the other minerals, present a few crystal faces, among which one as a cleavage-plane is prominent. The rest are very imperfect; and it is extremely difficult to get any measurements that are at all reliable from them. The goniometrical observations, however, were sufficient, together with the optical characters of the mineral, to determine that it belonged to the oblique system. These measurements gave the following approximate values:---

0 0 1, 1 0 0 = about 75° 30$^{\prime}$, those of diopside being 73° 59$^{\prime}$  
0 0 1, 1 1 0 = about 81°, those of diopside being 79° 29$^{\prime}$  
1 1 0, 1 0 0 = 45° 54$^{\prime}$ to 47° 26$^{\prime}$, those of diopside being 46° 27$^{\prime}$  
1 1 0, 1$^{\prime}$ 1 0 = 85° 8$^{\prime}$ to 86° 20$^{\prime}$, those of diopside being 87° 5$^{\prime}$  
1 0 0, 1 1 1 ? = 53° 25$^{\prime}$ to 54° 15$^{\prime}$, those of diopside being 53° 50$^{\prime}$  
0 0 1, 1$^{\prime}$ 1 0 = 100° 8$^{\prime}$, those of diopside being 100° 57$^{\prime}$

A slide cut for the microscope from a fragment of the nodule was found to exhibit a section of one of the crystals of this mineral cut very nearly parallel to the plane of symmetry. Two of the edges bounding this section were parallel, the one to a series of lines running through the crystal corresponding to its cleavage-planes, the other to certain bands that are constantly present in this augite, generally parallel to the plane 0 0 1, and formed of a white doubly refracting silicate, no doubt of the enstatite next to be described, intercalated in microscopic layers through the augite. These two edges represent the planes 1 0 0 and 0 0 1 as seen in a section nearly parallel to the plane of symmetry. They gave a normal angle of 0 0 1, 1 0 0 = 75° 15$^{\prime}$. In diopside this angle is 73° 59$^{\prime}$.

Light traversing this section of the crystal between crossed Nicols is at its maximum of extinction when polarised in a plane parallel or perpendicular to a line, making with 0 0 1 an angle very near to 22° 45$^{\prime}$, and with 1 0 0 an approximate angle of 52° 30$^{\prime}$. In diopside the second mean line makes corresponding angles of 22° 5$^{\prime}$ and 51° 6$^{\prime}$ with these normals. These measurements were made by an eyepiece goniometer fitted to the microscope, and having a fixed spider-line to indicate the plane of polarisation, while a rotating line is employed to measure on a graduated circle the inclinations of the edges and other directions in the section.

A section made parallel to the plane 1 0 0 exhibited one of the optic axes on the limit of the field of view in a Nörremberg's polariscope. The plane containing the optic axes is perpendicular to the edge [1 0 0, 0 0 1], and the optical character in the centre of the field is negative.

In all the above respects the mineral accords with diopside.

When looked through in a direction nearly normal to 0 0 1 or 0 1 0, or indeed in any direction parallel to the zone circle [0 0 1, 0 1 0], the crystals show a remarkable dichroism, which is, however, especially conspicuous when the direction is nearly normal to the plane 0 0 1.

If the section is parallel to the plane of symmetry, light polarised in a plane perpendicular to the principal section containing the acute mean line and the axis of symmetry is transmitted, of a pale pink lilac; when the crystal is turned 90°, so as to bring the same principal section into parallelism with the plane of polarisation, the transmitted tint is bluer, exhibiting a pale slate-blue or lavender.

The plane 1 0 0 presents a somewhat facile cleavage, much more readily obtained than cleavages which are also met with on the planes of the form 1 1 0, the latter being interrupted and uneven. The plane 1 0 0 is also conspicuous for a remarkable metallic lustre, recalling that seen on some kinds of diallage, but of a fine golden hue.

Two analyses of this mineral by the method already described, the silica being distilled as silicium difluoride and determined as potassium fluosilicate, gave the following results:---

|1.|2.|Mean.|Oxygen ratios.  
Silicic acid|55.389|55.594|55.491|29.28  
Magnesia|23.621|23.036|23.328|9.331  
Lime|20.02|19.942|19.981|5.709  
Iron oxide|0.78|0.309  
Soda|0.554|[0.554]  
Lithia|a trace|[a trace]

Viewed as a magnesium and calcium silicate the percentage composition becomes ---

|The formula (5/8Mg 3/8Ca)SiO\textsubscript{3} requires  
Silicic acid 56.165|56.604  
Magnesia 23.612|23.585  
Lime 20.223|19.811  

This does not accord with the analyses of the ordinary varieties of augite, in which the calcium is usually in excess of the magnesium.

It is, however, to be observed that a small deduction of the corresponding magnesium silicate (enstatite) has to be made by reason of the presence of the white mineral intercalated in layers along the direction parallel to the plane 0 0 1, and sometimes also to a second plane of the crystal. This mineral is doubtless the enstatite next to be described, and its presence would only modify the true formula of the augite by adding to the proportion of the magnesian constituent. The amount of one equivalent of enstatite to three of augite that this explanation would require, is more than microscopic observations would warrant; and it is probable that the augite itself is richer in magnesium than is usual in terrestrial augites.

The small amount of the oxide that in this augite corresponds to the ingredient of Osbornite that I identify with a titanoid metal, is met with in the precipitate by ammonia from the solution of the bases, and is included with the iron oxide in the above analyses.
\clearpage
\section{Enstatite as a Constituent of the Busti Meteorite.}
\paragraph{}
Besides the augitic mineral that has just been described, there is present in this meteorite another silicate which is in fact its most important ingredient. The augite is present in greatest quantity in the nodule that contains the calcium sulphide, though it is met with in smaller amount in the other parts of the meteorite. But associated with it everywhere, and otherwise forming the mass of the stone, is the mineral I have next to describe. As seen in a microscopic section, it presents the appearance of a number of more or less fissured crystals with different degrees of transparency, sometimes quite clear, sometimes nearly opaque, and with a more or less symmetrical polygonal outline. These crystals are imbedded in a magma of fine-grained silicate, through which a sort of irregular meshwork of an opaque white mineral is seen to ramify. When the ingredients are mechanically separated and examined, it is not difficult to distinguish what seem to be three different minerals. One is rare; it is colourless and transparent, and may be obtained in small splinters that have the appearance of being the result of a definite cleavage. The little planes thus obtained are too often merely divisional surfaces without crystallographic significance; and where they possess a more definite character, they present such rude faces that the values obtained for the angles can rarely be relied on. Another form of the mineral mass is that of a grey semi-transparent splintery mineral, the fragments being generally very composite. From these two varieties I failed in obtaining the measurements of an entire zone, the planes in which belonged to the same individual, and the attempt to cleave these minute individuals apart only serves to destroy them. The third form is that of a dark grey glistening crystalline substance tabular in form and very opaque. It presented cleavages indistinctly marking the faces of a prism, for which the mean of several measurements gave an angle of { 88° 35$^{\prime}$ / 91° 25$^{\prime}$ }; and to the planes (1 1 0) of this prism a dull face (0 0 1) is perpendicular, which seems in this case to be a second and less facile cleavage.

The results subjoined were obtained from seven selected fragments of the other forms of this mineral. They lack the important check which the polariscope affords; for the substance was usually too opaque for the use of this instrument, or else too composite to give any value to the results obtained with it. The fragments experimented upon were extremely minute and fragile, often breaking into powder while being mounted for the goniometer, and the angles are necessarily only approximate.

Found.|In Breitenbach enstatite.  
1 0 0, 1 1 0 about 46°|45° 52$^{\prime}$  
1 1 0, 1$^{\prime}$ 1 0 87° 10$^{\prime}$ to 88°|88° 15$^{\prime}$  
1 0 0, 1 0 1 41° 34$^{\prime}$|41° 12$^{\prime}$  
0 1 0, 0 1 1\footnote{A dubious plane.} about 40°|40° 21$^{\prime}$

The planes 1 0 0 and 1 1 0 are cleavages. In some cases, generally where the crystals are very composite, a cleavage seems to run parallel to a plane inclined at 73° to 74° to the face (1 0 0) and 90° to the face (0 1 0). As the forms of the mineral presenting this plane contain calcium, I have been uncertain whether to attribute the importance of this plane in certain specimens to an intermixture of augite with the enstatite. The plane 1 0 4 is also a conspicuous one on the crystals of enstatite in the Breitenbach meteorite, and the angle (74° 4$^{\prime}$) which it makes with the plane 1 0 0 in that mineral is very near that of the inclination (73° 59$^{\prime}$) of the planes 1 0 0 and 0 0 1 of diopside.

The chemical analysis of these three minerals shows that they are really enstatite under different aspects. Where the substance contains no lime it presents itself as a simply prismatic mineral, the dark grey tabular variety; where lime is present, though to the amount of less than 2 per cent., the crystalline structure becomes more complex, and it is far more difficult to obtain pieces in which it is sufficiently definite in character to allow of any measurements at all. It seems probable that the augite is in these cases blended in minute quantity, by a sort of tessellation, with the enstatite, somewhat as the enstatite is seen to be intercalated in narrow bands between layers of the augite already described, although the enstatite in the latter case is in much larger relative amount. But I have failed to obtain satisfactory proof of the actual presence of the augite from the optical characters of the sections of the mineral as seen in the microscope; though these frequently exhibit a structure in a high degree composite in its crystalline characters, the principal sections of the different parts of the mineral being in these cases disposed at all sorts of angles of mutual inclination. The analysis of these minerals yielded the following numbers:---

Dark grey tabular variety.

|Percentages.|Oxygen ratios.  
Silicic acid|57.597|30.718  
Magnesia|40.64|16.238  
Lime|---|---  
Iron oxide|1.438|  
Potash|0.394|  
Soda|0.906|  
|100.975  

Transparent white variety.

|Percentages.|Oxygen ratios.  
Silicic acid|58.437|31.166  
Magnesia|38.942|15.564  
Lime|1.677|0.479  
Iron oxide|1.177|  
Potash|0.332|  
Soda|0.357  
|100.922  

Semi-transparent grey variety.

1.

|Percentages.|Oxygen ratios.  
Silicic acid|57.037|30.419  
Magnesia|40.574|16.217  
Lime|2.294|0.655  
Iron oxide|0.867  
Soda|---|  
Potash|---|  
Lithia|---|  
|100.772

2.

|Percentages.|Oxygen ratios.  
Silicic acid|57.961|30.912  
Magnesia|39.026|15.598|  
Lime|1.524|0.435  
Iron oxide|0.154  
Soda|0.68  
Potash|0.569  
Lithia|---|  
|99.914

3.

|Percentages.|Oxygen ratios.  
Silicic acid|57.754|30.802  
Magnesia|38.397|15.347  
Lime|2.376|0.678  
Iron oxide|0.423  
Soda|0.657  
Potash|0.569  
Lithia|0.016  
|100.192

The greater part of the soda and part of the potash in these analyses, as in those of the augite, is certainly due to an impurity traceable to a minute amount of these bases contaminating the hydrochloric acid employed. The iron is present partly as metallic iron in a state of minutest subdivision, in small part also without doubt in combination in the magnesian silicate. In every case the bases are slightly in excess of the amount requisite for the formula of enstatite. It would seem highly probable, from the comparison of these with the known analyses and with such as I shall have to offer of other meteorites, that where in these bodies the conditions under which the rock was formed were such that the silicic acid was in excess of that required by the formula for enstatite, it has remained uncombined in the form of a crystallised silica with the specific gravity of \emph{fused} quartz; but that where the magnesium and other bases were in excess, a basic silicate with the formula of olivine absorbed the supplementary portion of these bases. Where calcium is present, it probably converts into an augite a portion of the materials that otherwise would go to constitute enstatite.

In none of the particular meteorites hitherto examined in the Museum Laboratory has a trace of alumina been found, though it has been carefully looked for, and consequently no felspathic ingredient has been detected in them.
\clearpage
\section{General Analysis of the Busti Meteorite.}
\paragraph{}
In order to determine approximately the proportions in which the different ingredient minerals were present in the meteorite, and to ascertain whether any other mineral had escaped detection, an analysis of the fragments and dust of the stone from the neighbourhood of the nodule containing the sulphides and the augite was made. The material employed was that obtained on cutting the meteorite by a dry wheel-saw, used to prevent the introduction of foreign substances. 1.874 grm. were taken for analysis. The sulphur was determined, as in the case of the Oldhamite, as sulphide of silver, and as separated by means of carbon disulphide. Heated with hydrogen chloride, and afterwards with potash, there was dissolved by those reagents and by the carbon disulphide 16.873 per cent., the residue being 83.127. The soluble part gave the results in column 1, the insoluble part those in column 2; the sulphur and sulphuric acid being supposed to be present as calcium sulphide and sulphate respectively.

1.

|Percentages.|Oxygen ratios.  
Calcium sulphate|0.442|  
Calcium monosulphide|4.133|  
Iron oxide|0.194|  
Silicic acid|6.514|3.474  
Lime|0.022|0.006  
Magnesia|5.055|2.02  
Potash|0.099|0.017  
Soda|0.118|0.03  
Lithia|---|---  
|16.577

2.

|Percentages.|Oxygen ratios.  
Iron oxide|0.891|  
Silicic acid|46.357|24.727  
Lime|12.375|3.535  
Magnesia|23.266|9.299
Potash|0.14|0.023  
Soda|0.455|0.117  
Lithia|0.019|0.01  
|83.503

The insoluble part in this analysis would correspond to a composition SiO\textsubscript{3}(Mg\textsubscript{3/4}Ca\textsubscript{1/4}), which, if we consider the calcium as being present as a constituent of the augite and the formula of this mineral to be SiO\textsubscript{3}(Mg\textsubscript{5/8}Ca\textsubscript{3/8}), will give for the insoluble silicate of the rock in the neighbourhood of the nodule a composition of two equivalents of augite to one of enstatite. As the analysis of the soluble portion showed that some of the above minerals had been dissolved, it was thought advisable to determine what and how much of them were rendered soluble by the action of hydrogen chloride in the cold. For this purpose some fresh material, selected partly from the neighbourhood of the nodule, partly from a portion of the meteorite consisting entirely of silicates, was submitted to the action of a mixture of one of hydrogen chloride and two of water for sixty-six hours at ordinary temperatures. Some sulphuretted hydrogen was given off and 4.419 per cent. dissolved. By a further treatment of the insoluble portion with soda 1.204 per cent. was removed from it. The acid had dissolved 0.501 per cent. silica, an amount of calcium corresponding to 1.285 per cent. of Oldhamite, 1.896 per cent. of magnesia, and 0.564 per cent. of iron. The oxygen ratios of the silica and magnesia dissolved are as 0.91 to 0.758, and show that of the 3.601 per cent. magnesian silicates extracted about 2.65 per cent. was olivine, the residue being enstatite.\footnote{In 1863 Mr. William Dancer analysed some of this powder from the cutting of this meteorite in the Laboratory of Professor Bunsen at Heidelberg.  
The results, which he was so good as to place in my hands, were as follow:---\\
SiO = 52.73\\
MgO = 37.22\\
FeO = 4.28\\
MnO = 0.01\\
CaO = 1.18\\
NiO = 0.78\\
CaSO\textsubscript{4} = 1.58\\
Ca\textsubscript{3}PO\textsubscript{8} = trace\\
CaCl = 0.01\\
NaS = 0.76\\
KO = trace\\
LiO = trace\\
HO = 0.92\\
Total = 99.47\\
``The mass of the stone,'' he says, ``is evidently a monosilicate of magnesia, lime, potash, and lithia; the iron and nickel existing in small particles as nickel-iron together with a small portion of manganese. The percentage of substance soluble in water is 1.03; this consists of sulphide of sodium, chloride of calcium, sulphate of lime, and traces of lithia and potash.'' Of course from such material as was in Mr. Dancer's hands it was impossible for him to separate the different minerals.}

A formula for the augite rather richer in lime would no doubt give a truer statement of the composition; but it is as impossible to separate the small amount of enstatite intercalated in the layers of the augite, as it is to distinguish and remove the latter mineral from the enstatite with which it appears in general to be so intimately blended.
\clearpage
\section{The Action of Acids on the Enstatite and Augite.}
\paragraph{}
As it appeared of some importance to determine the degree to which these meteoric minerals were soluble in the acids used for separating the silicates of a meteorite, and whether an olivinous constituent could be found in the Busti aerolite associated with the enstatite, or with some other silicate, the augite and the enstatite described in the previous sections were submitted to this solvent action. Alternately digested for many hours at 100° C. in strong hydrogen chloride, diluted with its volume of water and in caustic potash for ten or twelve hours to remove the separated silica, each of the three forms of enstatite proved to be acted upon; and the results in each case showed that the acid exercises simply a solvent action upon the mineral without separating it into two or more distinct silicates.

The subjoined Table records these experiments. The different degrees in which the acid dissolved the minerals in either case was due to the more or less complete character of the trituration to which the minerals had been subjected.

I deemed it desirable in one case (and I selected the transparent variety for this purpose) to repeat the process three times so as to remove any doubt as to the nature of the action exercised by the acid.

Of the greyish-white variety, 1.0686 grm. was submitted to the action of acid for twenty hours at 100°, and subsequently to that of potash, to remove separated silica, for twelve hours. It yielded

Mineral dissolved = 0.1006 grm., \emph{viz.} {in acid, 0.0475 = 4.445 per cent.; in potash, 0.0531 = 4.969} = 9.414 per cent.

Mineral unacted on = 0.968 = 90.586 per cent. 

The analysis of the 9.414 per cent. dissolved is given in column 1 of the Table.

Of the grey tabular variety of enstatite 0.1478 grm. were treated by hydrogen chloride for sixteen hours at 100°, and subsequently by potash for a similar time.

The dissolved portion was 0.0115, \emph{viz.} {by acid, 0.0047 = 3.179 per cent.; by potash, 0.0068 = 4.6 per cent.} = 7.779 per cent.

The residue unacted on was 0.1363 = 92.19 per cent.

The 7.779 per cent. dissolved gave on analysis numbers the approximate composition of which is given in column 2.

Of the white variety of the enstatite 0.2082 grm. yielded on a \emph{first treatment} for twenty hours with acid, and subsequently with potash,

Dissolved mineral = 0.0264, \emph{viz.} {by acid, 0.0164 = 7.877 per cent.; by potash, 0.01 = 4.803 per cent.} = 12.68 per cent.

and undissolved = 0.1818 = 87.319 per cent.

The approximate composition of the 12.68 of dissolved mineral is given in column 3.

On a \emph{second} treatment of the undissolved portion, whereof after two hours further trituration 0.1674 grm. were operated on as before with acid for thirty, and with potash for twelve hours,

The mineral dissolved = 0.1137 \emph{viz.} {by acid, 0.0437 = 41.766 per cent.; by potash, 0.07 = 26.074 per cent.} = 67.84 per cent.

Mineral unacted on = 0.0539 = 32.16 per cent.

On a \emph{third} treatment in a similar way 0.0424 grm. yielded:

Of mineral dissolved = 0.0217, \emph{viz.} {by acid, 0.0091 = 21.47 per cent.; by potash, 0.0126 = 29.71} = 51.18

Of mineral unacted on = 0.0207 = 48.82 per cent.

|1.|2.|3.  
Silicic acid|5.408|5.141|6.724  
Magnesia|2.367|1.353|4.61  
Lime|1.048|0.270|0.432  
Iron oxide|0.187|0.676|0.576  
Potash|0.121|0.528|0.504  
Lithia|---|trace|trace  
Total|9.131|7.968|12.846  
Soda found|0.126|1.217|1.042

In the last two treatments of the white silicate the quantities, 0.0437 and 0.0091 grm., of ingredients dissolved by the acid and 0.07 and 0.0126 of silicic acid dissolved by the potash were too small for even approximate analysis. The ratio of silicic acid to the bases, neglecting the small amount of the former dissolved by the acid, is in the last case in the ratio of 58.4 SiO\textsubscript{2} : 42 bases, that of the analysis of this white variety giving a ratio of 58.4 : 41.6.

The degree to which the augite is soluble was determined by subjecting this mineral to a treatment similar to that by which the enstatite was dissolved; 0.2714 grm. so treated for eighteen hours by acid, and a similar time by potash at 100° C., gave 0.2614 of unchanged, and 0.0193 of dissolved mineral. This corresponds to 7.384 per cent. of the latter and 92.616 of the former.

There can be little doubt from these results that the action of acid on the minerals with the formula of enstatite or of augite is that simply of a solvent.
\clearpage
\section{The Iron of the Busti Meteorite.}
\paragraph{}
A small pepita of the iron, weighing 0.1997 grm., contained in the meteorite was analysed. A small quantity of silicates and of glistening Schreibersite was left at first undissolved by hydrogen chloride. A second treatment with acid dissolved the latter.

The first solution contained a trace of phosphoric acid, and a small quantity of hydrogen sulphide came off from the iron during its solution. The metallic constituents of the dissolved portion were separately determined, and an analysis was also made of the Schreibersite. The results were:

Silicates 16.725 per cent.  
Schreibersite:  

Iron = 0.736  
Nickel = 0.195  
Phosphorus = 0.07  
1.001  

Iron 79.069  
Nickel 3.205  
100.000

or, omitting the silicate,

Iron-nick alloy {Iron 94.949, Nickel 3.849} --- 98.798  
Schreibersite {Iron 0.884, Nickel 0.234, Phosphorus 0.084} --- 1.202  
100.000

The quantity was far too small to encourage a search for cobalt and other metals.

Besides the nickeliferous iron, which is disseminated very sparsely, and in particles singularly unequal as regards their size and distribution, and with which troilite is associated in very small quantity, chromite is present as a constituent of small but appreciable amount.

The crystals of this mineral are distinct, and sometimes present minute brilliant faces and good angles for measurement. One of these gave the solid angle of a regular octahedron.
\clearpage
\section{11. On the Manegaum Meteorite of 1843.}
\paragraph{}
Of the circumstances attending the fall of the Manegaum meteorite, and of its appearance as seen in section in the microscope, I gave some account in the Philosophical Magazine of 1863.

I was precluded at that time from making chemical analyses at the British Museum, and was unable to investigate with any precision the nature of this stone, which is one of those that, in the well-defined character of a chief ingredient, offers considerable advantages for the inquiry on which I am engaged.

The meteorite fell on the 29\textsuperscript{th} of June,\footnote{The date usually assigned to this fall, \emph{viz.} the 16\textsuperscript{th} of July, is erroneous. The true day of the fall is given, in the Mahratta account of it, as the third day of the month Asarh sudi, on Thursday. I am indebted to General Cunningham for the identification of this date with the 29\textsuperscript{th} of June, 1843. [For the account see Journal As. Soc. Bengal, vol. 13. p. 880.]} 1843, at Manegaum in Khandeish, at half past three o'clock PM. A very small part of it was preserved; and of this a little fragment was sent to the British Museum, as one among many valuable contributions of the kind from the Asiatic Society of Bengal.

The examination and the analysis of this meteorite had therefore to be performed on very minute quantities; in fact on a few grains of \emph{débris} that had become detached from the brittle little stone.

The conspicuous ingredient in this meteorite is a pale yellow-green, or primrose-coloured mineral, with a tint similar to that of a very pale peridot or chrysolite, occurring in crystalline grains cemented together, in a state of very slight aggregation, by a white opaque silicate, which in a microscopic section has a flocculent appearance.

The granules of green minerals present in the microscope the appearance of tolerably symmetrical crystals, and are seen of every size, from that of a small pin's head to that of a microscopic dust.

In separating this green mineral from the fragile mass, I have never succeeded in obtaining a crystal of it entire.

The mineral is enstatite; if, at least, we are to include under this name every isomorphous mixture of iron and magnesium silicates with the formula MSiO\textsubscript{3} and crystallised in the prismatic system, but without the distinguishing features either of hypersthene or of diaclasite.

Two of the grains selected from the picked green mineral for measurement gave the following results, ---

Manegaum enstatite.|Breitenbach enstatite.  
1 0 0, 1 1 0 = about 46°|45° 52$^{\prime}$  
1 0 0, 1 0 1 = 49° 4$^{\prime}$|48° 49$^{\prime}$  
1 1 0, 1$^{\prime}$ 1 0 = about 88°|88° 16$^{\prime}$  
1 1 0, 1 0 1 = 58° 39$^{\prime}$|58° 24$^{\prime}$  

The comparison of these measurements with those obtained from the enstatite of the Breitenbach siderolite given in the second column will suffice for the identification of the two minerals as enstatite.

The analysis of the Manegaum mineral was performed by the method of distillation already described. 0.2658 grm. was taken.

|per cent.|Oxygen ratios.  
Silicic acid = 0.14805|55.699|29.706  
Magnesia = 0.0606|22.799|9.119  
Iron monoxide = 0.0546|20.541|4.564  
Lime = 0.0035|1.316|0.376  
0.26675|100.355|  

If we allow for the probable presence of a little augite, corresponding proportionately to the lime found in the analysis, this Manegaum mineral will have the formula of an enstatite richer in iron that even that of the Breitenbach siderolite, the formula for which is (4/5 Mg 1/5 Fe) SiO\textsubscript{3}. The formula (2/3 Mg 1/3 Fe) SiO\textsubscript{3} requires a percentage composition of SiO\textsubscript{2} = 54.2; MgO = 24; FeO = 21.7, which would accord very closely with that of the Manegaum enstatite if we deduct the 1.5 per cent. of silica that the analysis gives in excess.

The specific gravity of the Manegaum enstatite is 3.198, its hardness is 5-6.

A small portion of the meteorite was taken for analysis in its entirety. A black mineral disseminated in a band running through it in minute crystalline particles is chromite; its formula is assumed as FeCr\textsubscript{2}O\textsubscript{4}. 0.4078 grm. was analysed by the hydrogen fluoride method, and gave the following results:---

|per cent.|Oxygen ratios.  
Silicic acid = 0.2187|53.629|28.602  
Magnesia = 0.0951|23.32|9.328  
Iron monoxide = 0.0835|20.476|4.55  
Iron monoxide = 0.0013||  
Chromium sesquioxide = 0.0029||  
Lime = 0.0061|1.495|0.427  
0.4076|99.949  

The silicic acid in this sample of the entire meteorite is in the exact proportion requisite for the enstatite formula; it is therefore not improbable that the excess found in the green enstatite may have been due to an error in the analysis rather than to the presence of either free silica or of a silicate with a higher proportion of this ingredient.

The Manegaum meteorite contains a very minute amount of meteoric iron, far too small for isolation and analysis; indeed the portion taken for analysis could hardly have contained a trace of it.

This meteorite is interesting as presenting us with an instance of a meteoric rock constituted of a single silicate, and that enstatite. It differs from the mass of the Busti meteorite in that the latter is a nearly pure magnesian enstatite, while that of Manegaum is a highly ferriferous one. The two meteorites concur also in the light they throw on the nature of the flocculent opaque white mineral seen in the microscopic sections of many meteorites. In these two cases, at least, that mineral is enstatite.

\centerline{*\hspace{15mm}*\hspace{15mm}*\hspace{15mm}*\hspace{15mm}*}
\bigskip

In concluding this memoir, in which I have endeavoured to deal as exhaustively as possible with the constitution and characters of two remarkable meteorites, I wish to record the great services rendered me in its investigation by Dr. Flight, Assistant in my Department at the British Museum, to whose manipulatory skill and care I am greatly indebted in the chemical part of the inquiry.
\clearpage
\section{Explanation of the Plates.}
\subsection{Plate 1.}
\paragraph{}
Figure 1 and Figure 2 represent sections of crystals of the augite as seen in the microscope with a power of 45 linear; that in fig. 1 is nearly in the zone [1 0 0, 0 1 0] and a little oblique to the plane 1 0 0; that in fig. 2 is slightly oblique to the plane 0 0 1, and a little so also to the zone [0 0 1, 0 1 0]. The lines P, P indicate the planes of vibration.

Figure 3 and Figure 4 represent sections, as seen in a microscopic slide cut from the meteorite, of crystals of enstatite, that in fig. 3 being nearly in the zone of the prism planes, and that in fig. 4 being nearly perpendicular to these.
\subsection{Plate 2.}
\paragraph{}
Representation of the Busti meteoric stone, in two views taken from opposite points. The letter N indicates the position of the nodule containing the Oldhamite and Osbornite, of which a section is also represented in the Plate. All the drawings are of the natural size.
\clearpage
\pagestyle{fancy}
\fancyhf{}
\rhead{Plate 1.}
\cfoot{\thepage}
\begin{figure}[H]
\centering
\includegraphics[height=90mm,keepaspectratio]{Plate1-Figure1.png}
\caption{\small Figure 1.}
\end{figure}

\begin{figure}[H]
\centering
\includegraphics[height=90mm,keepaspectratio]{Plate1-Figure2.png}
\caption{\small Figure 2.}
\end{figure}

\begin{figure}[H]
\centering
\includegraphics[height=75mm,keepaspectratio]{Plate1-Figure3.png}
\caption{\small Figure 3.}
\end{figure}

\begin{figure}[H]
\centering
\includegraphics[height=75mm,keepaspectratio]{Plate1-Figure4.png}
\caption{\small Figure 4.}
\end{figure}
\clearpage
\rhead{Plate 2.}
\begin{figure}[H]
\centering
\includegraphics[width=\textwidth,keepaspectratio]{Plate2.png}
\caption{\small Plate 2.}
\end{figure}
\end{document}
